\documentclass[12pt]{article}

\title{Geo 2 tag LV Demo (draft)}

\begin{document}
\maketitle          

\section{Feature descriptions}

This section enumerates and describes features which implemented
for LV demo. 

\paragraph{View map with marks}

User can select view part of google map on his device. Map shifting and scaling
are implemented. For each mark there is a bubble with one latin letter from 
A to Z. There is a limit to number marks on picture at the same time.

\paragraph{View marks as text feed}

User has a table with next columns: date, mark letter, tag text. He could see text marks
with nearest (we have parameter to define what does it mean) location base.

\paragraph{Getting GPS/location data form device}

Application uses n900 GPS location interface to determine current location and
receive correct marks and map pieces. 

\paragraph{Persist marks on remote server}

Each mark are placed on remoe server in the database and has next structure:
\begin{itemize}
  \item timestamp
  \item position: latitude, longitude
  \item text message
\end{itemize}

Aditionally each mark usually belongs one or several channels.

\paragraph{Select channels }

User has interface for selech channels which he is interested in. It's checkboxed list
with channel names.

\paragraph{Add mark}

User has dialog for adding new text message to selected channel. Time and position will
be associated with text automaticaly.

\section{Not implemented features}

This section enumerates features which wonn't be implemented for LV demo.

\paragraph{Add pictures and other MM content}

Current client an server implementation aren't support multimedia content like picture,
voice and video. Only text messages are supported. We have several raw parts to support 
it in the future.

\paragraph{User profiles and security}

We do not support limitation for access to channels by user name. Actually we have only
one use who could setup channels set for getting/setting marks.

\paragraph{Google marks integration}

We don't use (do not export/import) original google marks which can be setup in browser.
But we can. It should be supported in next versions.

\paragraph{Web interface for tag server}

We have not user interface to the server. Actually we have to implement it as far as possible.

\end{document}   

